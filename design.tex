\documentclass[11pt]{article}

\title{PID Controller Design}
\author{Michael Adams}
\date{May 3, 2015}

\begin{document}
\maketitle

\section{Introduction}

This document describes the hardware and software design of a general
purpose, multichannel, digital PID control system. The system supports
eight concurrent locking channels with ADC inputs and user-configurable
DAC or DDS outputs. Each channel operates at a maximum rate of 100kHz.
Optional digital filtering stages may be activated at the cost of
reduced operational rates. Output processing stages enforce output
bounds and support optional output scaling. A custom Python graphical
interface allows for real time configuration of controller parameters
and monitoring of controller state.

Controller logic is described in Verilog and implemented on a Xilinx
Spartan 6 FPGA\@. The FPGA is mounted on a circuit board that contains
an 8-channel ADC (Analog Devices AD7608, 18-bit resolution, 200 ksps)
and an 8-channel DAC (Texas Instruments DAC8568, 16-bit resolution). The
board provides I/O ports for interfacing with external DDS chips.

A high level block diagram of the system is shown below in Fig. 1.
Up to eight analog process variable signals are probed by the AD7608
chip. Data is sampled at a rate of 200kHz and passed through a digital
first-order sinc filter on the AD7608. The filter oversamples at a rate of
2x. The digitized and filtered signals are read by the ADC Controller
module at a rate of 100kHz over two serial interfaces. The ADC
Controller passes data into the PID pipeline, which computes the PID
sum. Data is finally routed to DDS or DAC controllers, depending on
configuration options. System parameters are set and state probed
through the FrontPanel Controller, which communicated via USB 2.0 link
with the Python GUI\@.

\section{Hardware Design}

Fig. 1 shows a block diagram of the controller hardware layout. Data
flows from left to right. The ADC Controller interfaces with the AD9912
chip, reading data over two serial interfaces at a rate of 100kHz. 

\subsection{Overview}

\subsection{ADC Controller}

\subsection{Clock Synchronizer}

\subsection{Moving Average Filter}

\subsection{Router}

\subsection{Output Processor}

\subsection{Frontpanel Controller}


\section{Software Design}

\end{document}


